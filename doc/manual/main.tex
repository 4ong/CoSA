\documentclass{article}
\usepackage[margin=1.0in]{geometry}
\usepackage{graphicx}
\usepackage{paralist}
\usepackage{listings}
\usepackage{hyperref}
\usepackage{amsthm}
\usepackage{xcolor}

\theoremstyle{definition}
\newtheorem{example}{Example}[section]
\renewcommand\lstlistingname{Figure}

\lstdefinelanguage{ets}
{
morekeywords={->,True, False},%
morecomment=[l]\#,%
commentstyle=\color{red}\ttfamily,
%backgroundcolor=\color{gray!10},
keywordstyle=\color{blue}\bfseries,
numbers=left,
basicstyle=\footnotesize\ttfamily,
numberstyle=\small\ttfamily,
numbersep=1.5em,
showstringspaces=false,
breaklines=false,
frame=lines,
%framexleftmargin=2.5em,
xleftmargin=2.5em,
}

\lstdefinelanguage{sts}
{
morekeywords={->,True, False, VAR, OUTPUT, TRANS, DEF, INIT, INVAR},%
morecomment=[l]\#,%
commentstyle=\color{red}\ttfamily,
%backgroundcolor=\color{gray!10},
keywordstyle=\color{blue}\bfseries,
numbers=left,
basicstyle=\footnotesize\ttfamily,
numberstyle=\small\ttfamily,
numbersep=1.5em,
showstringspaces=false,
breaklines=false,
frame=lines,
%framexleftmargin=2.5em,
xleftmargin=2.5em,
}

\begin{document}

\title{\textbf{CoSA}\\Manual}
\author{Cristian Mattarei}

\maketitle

\newpage
\tableofcontents

\newpage
\section*{Introduction}

CoSA is a symbolic model checker for hardware design. It incorporates
a variety of state-of-the-art techniques to achieve performance, while
providing a simple and intuitive interface. This document describes
all functionalities provided by CoSA, including also a series of
running examples. 


\section{Overview}

The main inputs to CoSA to define a model checking
(\textsection~\ref{sec:model_checking}) verification task are:
\begin{itemize}
\item a list of comma-separated input files
  (\textsection~\ref{sec:input_formats}) describing the hardware;
\item a verification problem
  (\textsection~\ref{sec:problem_definition}) e.g.; safety
  (\textsection~\ref{sec:safety}), LTL (\textsection~\ref{sec:ltl}),
  or equivalence checking (\textsection~\ref{sec:equivalence}); and
\item a property (\textsection~\ref{sec:properties}).
\end{itemize}

\noindent
The other parameters can be divided into:
\begin{itemize}
\item encoding options (\textsection~\ref{sec:encodings});
\item performance optimizations parameters
  (\textsection~\ref{sec:optimizations}); and
\item debugging (\textsection~\ref{sec:debugging}).
\end{itemize}

\noindent
The results of the analyses, as counterexample traces, are provided in
multiple formats (\textsection~\ref{sec:results_analysis}).

\

\noindent
All these options can be either provided as parameters to CoSA on the
command line, or as a single problem file
(\textsection~\ref{sec:problem_file}).

\

\noindent
For more information on the actual parameters, run CoSA with
\texttt{-h}.


\section{Background}

\subsection{Model checking}
\label{sec:model_checking}
Model checking is a technology that allows for an efficient and
exhaustive testing of a system. Model checking can be applied to
different domains, and the common problem this technique can solve is
to prove that a system - in our case a hardware design - meets a set
of predefined and expected behaviors, usually represented as system
assertions. A model checking problem is usually denoted as $M \models
\varphi$, where $M$ is a mathematical representation of the system,
and $\varphi$ is the expected behavior. For instance, given a hardware
component \emph{Sum} that computes the sum of its input ports $I_1$
and $I_2$, and that provides the result to the output $O$, a
conventional testing procedure for \emph{Sum} would require to check
that if $I_1=0$ and $I_2=0$ then $O=0$, and that $I_1=1$ and $I_2=0$
results into $O=1$, and so on. However, with model checking we can
directly evaluate if \emph{Sum} $ \models (O=I_1+I_2)$, and if this is
not the case the model checker (such as CoSA) will provide a
counterexample to the expected behavior (i.e., $O=I_1+I_2$), which is
represented as a series of assignments to the ports such as $I_1=16,
I_2=13, O=32$.

Model checking is a fundamental technology when developing complex
systems, thus over the years multiple different techniques have been
developed to efficiently solve the problem. A major distinction is
between symbolic and explicit state (model checking), and the
different resides in the technique used to represent the system. 

\subsection{Symbolic transition system}
\label{sec:sts}

todo: describe also synchronous product


\subsection{Linear Temporal Logic (LTL)}
\label{sec:ltl}

\subsection{Safety verification}
\label{sec:safety}

\subsection{Equivalence checking}
\label{sec:equivalence}


\section{Input formats}
\label{sec:input_formats}

\subsection{System description}

CoSA supports multiple input formats, and while being equivalent from
an expressivity point of view, they are tailored and design to
accomplish different purposes. CoSA distinguish between different
formats by relying on their file extension. Running \texttt{CoSA -h}
shows the list of accepted input formats, and make sure that the
configuration is correct.

CoSA accepts a comma-separated list of files, and the resulting model
is \textbf{synchronous product} (\textsection~\ref{sec:sts}) between
all of them. This allows for a clear and explicit definition of
\textbf{reset procedures}, as well as environmental assumptions.

An input file can be provided with additional model flags listed in
squared brackets, which instruct the encoder of how to process the
file.

\begin{example}[Model flags]
   \texttt{CoSA -i input\_file.extension[model\_flag]} provides
   \texttt{input\_file.extension} as input file, with
   \texttt{model\_flag} as a model flag.
\end{example}


Following, we cover all the formats supported by CoSA.

\subsection{Verilog}
This format is natively supported by relying on PyVerilog, and it
requires a top-level model flag.

\begin{example}
  \texttt{CoSA -i examples/counters\_4/counters\_4.v[Counters\_4]}
  provides \texttt{examples/counters\_4/} \texttt{counters\_4.v} as input model,
  with \texttt{Counters\_4} as a top module.
\end{example}

\subsection{SystemVerilog}
SystemVerilog is supported using Verific, which is an industrial tool
that can translate SystemVerilog to Verilog. After translating the
file using Verific, CoSA relies on the internal Verilog encoder to
process the model. The SystemVerilog encoder also requires to provide
the top module as model flag.

\begin{example}
  \texttt{CoSA -i examples/counters\_4/counters\_4.sv[Counters\_4]}
  provides \texttt{examples/counters\_4/} \texttt{counters\_4.sv} as input model,
  with \texttt{Counters\_4} as a top module.
\end{example}

\subsection{CoreIR}
CoreIR~\cite{CoreIR} is supported by relying on PyCoreIR. Its file
extension is \texttt{.json}. The format accepts model flags to
instruct the encoder to extract additional information from the model
such as lemmas from CoreIR optimization passes \cite{cosa-paper}.

Currently, CoSA supports the following CoreIR model flags:
\begin{itemize}
\item \texttt{FC-LEMMAS}: it automatically extracts the lemmas from
  the fold constants CoreIR pass.
\end{itemize}

\begin{example}
  \texttt{CoSA -i examples/counters/counters.json} to load the
  \texttt{examples/counters/counters.json} design, or \texttt{CoSA -i
    examples/fold-constants/mpe\_fc.json[FC-LEMMAS]} to load
  \texttt{examples/fold-constants/} \texttt{mpe\_fc.json} with
  \texttt{FC-LEMMAS} as model flag.
\end{example}

\subsection{BTOR2}
BTOR2~\cite{btormc} is a very concise format to represent SMT-based
hardware designs. Its extension is \texttt{.btor} or \texttt{.btor2},
and it is mainly used to interface with other tools such as
Yosys~\cite{wolf2013yosys}, since it can produce such format.

\subsection{Symbolic Transition System (STS)}

This format allows for the definition of a component-based Symbolic
Transition System, which is characterized by:

\begin{itemize}
\item system variables, divided into \texttt{STATE}, \texttt{INPUT},
  \texttt{OUTPUT}, and \texttt{VAR}
\item initial states formula, i.e., \texttt{INIT}
\item transition relation formula, i.e., \texttt{TRANS}
\item invariant formula, i.e., \texttt{INVAR}, which constraints every
  states of the system, including the initial ones
\end{itemize}

The language supports also a typed modules definition and
instantiation. A module can be defined using the keyword \texttt{DEF}, followed
by a list of parameters, while its instantiation should be defined in
the \texttt{VAR} section.

Simple definition of an 8-bit counter with clock and reset is reported
in Figures \ref{counter-sts}, and \ref{counterh-sts}. The latter is
defined using sub-modules instantiation, and the lines that starts
with \texttt{\#} are comments.

\begin{lstlisting}[frame=single,language=sts,caption=Counter example,label=counter-sts]
VAR
clk: BV(1);
rst: BV(1);

OUTPUT
out: BV(8);

INIT
out = 0_8;
clk = 0_1;

TRANS
# Clock behavior definition
(clk = 0_1) <-> (next(clk) = 1_1);
# When posedge and not reset we increase out by 1
(posedge(clk) & ! posedge(rst)) -> (next(out) = (out + 1_8));
# When not posedge and not reset we keep the value of the out
(! posedge(clk) & ! posedge(rst)) -> (next(out) = (out));
# When reset we set out to 0
posedge(rst) -> (next(out) = 0);
\end{lstlisting}


\begin{lstlisting}[frame=single,language=sts,caption=Counter example (hierarchical),label=counterh-sts]
VAR
  clk: BV(1);
  rst: BV(1);
  counter_1: Counter(clk, rst);

OUTPUT
  out: BV(8);

TRANS
  # Clock behavior definition
  (clk = 0_1) <-> (next(clk) = 1_1);

INVAR
  # Enforcement of the equivalence between local output value
  # and the output of the sub-module
  out = counter_1.out;

DEF Counter(clk: BV(1), rst: BV(1)):
  VAR
  out: BV(8);

  INIT
  out = 0_8;

  TRANS
  # When posedge and not reset we increase out by 1
  (posedge(clk) & ! posedge(rst)) -> (next(out) = (out + 1_8));
  # When not posedge and not reset we keep the value of the out
  (! posedge(clk) & ! posedge(rst)) -> (next(out) = (out));
  # When reset we set out to 0
  posedge(rst) -> (next(out) = 0);
\end{lstlisting}


\subsection{Explicit state Transition System (ETS)}

This format allows for the definition of an Explicit States Transition System, which is characterized by two sections:
\begin{itemize}
\item states definition, expressed as values assignments to system
  variables, e.g., \texttt{I: clk = 0\_1} for the initial state or
  \texttt{S1: output = 4\_8} for the state S1, and so on
\item transitions definition, e.g., \texttt{I -> S1} defines a
  transition from the state I to the state S1
\end{itemize}

The language does not require types definitions, because they are
inferred by the values assignment. Figure~\ref{counter-ets} shows a an
example of a 2-bit counter, and the lines that starts with \texttt{\#}
are comments.

\begin{lstlisting}[frame=single,language=ets,caption=2-bit Counter,label=counter-ets]
# States definition
I: output = 0_2
S1: output = 1_2
S2: output = 2_2
S3: output = 3_2

# Transitions
I -> S1
S1 -> S2
S2 -> S3
S3 -> I
\end{lstlisting}

The ETS format is particularly suited for the definition of sequential
behaviors such as the \textbf{reset procedures}. In fact, most
hardware definitions require to be properly initialized before
performing any analysis. In the following example, the
\texttt{reset\_done} variable is used to keep track of the reset
status, and it is used to specify a pre-condition for the verification
properties.

\begin{lstlisting}[frame=single,language=ets,caption=Reset procedure example,label=reset]
I: rst = 0_1
I: reset_done = False

S1: rst = 1_1
S1: reset_done = False

SE: rst = 0_1
SE: reset_done = True

I -> S1
S1 -> SE
# the reset_done signal remains up forever, defined as a self-loop on the SE state
SE -> SE
\end{lstlisting}

\begin{example}
  \texttt{CoSA -i
    examples/counters\_4.v[Counters\_4],examples/counters\_4/rst\_beh.ets}
  to load the \texttt{examples/counters\_4.v} design and the ETS
  \texttt{examples/counters\_4/rst\_beh.ets} and perform a synchronous
  product between them.
\end{example}

\section{Properties}
\label{sec:properties}

CoSA supports the definition of invariant and Linear Temporal Logic
(LTL) properties. Moreover, the tool allows for the definition of
syntactic sugar and parametric generators. CoSA relies on
PySMT~\cite{gario2015pysmt} for the formulae management, thus the
syntax is going to related to this tool.

\subsection{Invariant}
An invariant property is a propositional formula over the system
variables that has to hold at any time. 

\begin{example}
  Given the \texttt{examples/counters\_4.v} model, the invariant property
  \texttt{out < 10\_16} states that the value of \texttt{out} should
  be always less than the decimal encoding of \texttt{10} as a
  bit-vector of size {16}.
\end{example}

An invariant property can predicate also over the \texttt{next}
variables, which are the primed version of each variable.

\begin{example}
  Given the \texttt{examples/counters\_4.v} model, the invariant property
  \texttt{(rst = 1\_1) -> (next(out) = 0\_16)} states that when the reset
  signal \texttt{rst} is equal to 1 (encoded as a bit-vector of size
  {1}), the primed value of \texttt{out} should be \texttt{0} (encoded
  as a bit-vector of size {16}).
\end{example}

\subsection{Linear Temporal Logic}
The support for LTL extends propositional logic with additional
temporal operators:
\begin{itemize}
\item \texttt{G}: globally, e.g., $G(\varphi)$ means that $\varphi$
  has to hold for every states;
\item \texttt{F}: finally or eventually, e.g., $F(\varphi)$ means that $\varphi$
  has to eventually hold;
\item \texttt{X}: next, e.g., $X(\varphi)$ means that $\varphi$
  has to hold at the next state;
\item \texttt{U}: until, e.g., $\gamma U \varphi$ means that $\gamma$
  has to hold until $\varphi$ becomes \emph{True};
\item \texttt{R}: release, e.g., $\gamma R \varphi$ means that $\varphi$
  is \emph{True} until $\gamma$ becomes \emph{True}.
\end{itemize}

\begin{example}
  Given the \texttt{examples/counters.json} model, the LTL property
  \texttt{F(self.out = 4\_16)} states that the signal
  \texttt{self.out} must reach the value \texttt{4} at some time in
  the future.
\end{example}

\begin{example}
  Given the \texttt{examples/counters.json} model, the LTL property
  \texttt{F(G(self.out = 4\_16))} states that the signal
  \texttt{self.out} must reach the value \texttt{4} at some time in
  the future, and keep the value forever.
\end{example}

\begin{example}
  Given the \texttt{examples/counters.json} model, the LTL property
  \texttt{G(F(self.out = 4\_16))} states that the signal
  \texttt{self.out} must reach the value \texttt{4} infinitely many
  times.
\end{example}


\subsection{Syntactic sugar}

In order to simplify the definition of hardware properties, CoSA
integrates an expandable support for syntactic sugar. Additional
syntactic sugars should be defined in \texttt{cosa/encoders/sugar.py},
and registered in \texttt{cosa/encoders/template.py}. The one that are
registered can be seen in the \texttt{special operators} section of
the CoSA helper (parameter \texttt{-h}). Currently, CoSA supports the
following syntactic sugar:

\begin{itemize}
\item \texttt{posedge}, with a single
  parameter. \texttt{posedge(variable)} is equivalent to \texttt{(variable
    = 0\_1) \& next(variable = 1\_1)}, if \texttt{variable} is a
  bit-vector of size 1, while \texttt{(variable) \& next(! variable)} if
  \texttt{variable} is of Boolean type;
\item \texttt{negedge}, which is implemented as \texttt{posedge} with
  inverted polarity;
\item \texttt{change}, with a single
  parameter. \texttt{change(variable)} is equivalent to
  \texttt{(variable != next(variable)};
\item \texttt{nochange}, with a single
  parameter. \texttt{nochange(variable)} is equivalent to
  \texttt{(variable = next(variable)};
\item \texttt{ones}, with a single parameter. \texttt{ones(variable)}
  provides the highest possible value of \texttt{variable};
\item \texttt{zero}, with a single parameter. \texttt{zero(variable)}
  provides the bit-vector encoding of all zeros, given the size of
  \texttt{variable};
\item \texttt{dec2bv}, with a two parameters. \texttt{dec2bv(value,
  variable)} provides the bit-vector encoding of value, given the size
  of \texttt{variable}.
\end{itemize}

\subsection{Generators}

More complex behaviors sometimes require the definition of parametric
modules to support the verification. The generators help on
simplifying those properties, by providing a parametric way to defined
commonly used symbolic transition systems. A common example of a
generator is the definition of a \emph{scoreboard} to verify the
behavior of a \emph{FIFO}. The purpose of a \emph{scoreboard} is to
keep track of a packet that goes into the \emph{FIFO}, and tells when
it is supposed to come out, according with the number of pushes and
pops that have been performed. An approach to define such behavior
would be to define an STS that models the \emph{scoreboard}, which in
synchronous product with the original system would allow the user to
specify the appropriate property. However, with the CoSA generators
this verification is much simpler.

\begin{example}
  Given the FIFO implementation in \texttt{examples/fifo/fifo.sts},
  the property that verify its behavior is defined as \texttt{sb.end
    -> (sb.packet = output)} by instantiating the generator
  \texttt{sb=FixedScoreboard} \texttt{(input, 6, posedge(clk))}. The meaning of
  the property is that when signal \texttt{end} from the
  \emph{scoreboard} \texttt{sb} is \emph{True}, then the
  \texttt{packet} should be equivalent to the \texttt{output} of the
  \emph{FIFO}. While the instantiation of the \emph{scoreboard} takes
  as input the \texttt{input} of the \emph{FIFO}, its length (in this
  case is 6), and the signal that triggers the coming of a new packet.
\end{example}

The generators can be defined in \texttt{cosa/encoders/generators.py}
and should be registered in \texttt{cosa/
  encoders/template.py}. CoSA
currently provides this set of generators:

\begin{itemize}
\item \texttt{FixedScoreboard}: \emph{scoreboard} for a FIFO with no
  pop. The parameters are (input\_port, max\_value, push\_signal);
\item \texttt{Scoreboard}: \emph{scoreboard} for a FIFO with push and
  pop. The parameters are (input\_port, max\_value, push\_signal,
  pop\_signal);
\item \texttt{Random}: provides a non-deterministic value. The
  parameter is (size), and it can be either a variable (it would
  consider its size), or a decimal value.
\end{itemize}  

\section{Verification problems}
\label{sec:problem_definition}


\subsection{Environmental assumptions}

\subsection{Simulation}
The simulation generates a system execution of a provided depth. This
verification does not require a property, in that case it will be set
to \emph{True}. If a property is provided the shortest execution
reaching a state that satisfies the property is provided. If such a
state does not exist, the verification
fails. Table~\ref{tab:simulation_results} summarizes the possible
results of the analysis, given \texttt{bmc\_length} parameter.

\begin{table}[h]
  \centering
\begin{tabular}{ c c | c c }
  Property & Trace exists? & Result & Trace length \\ \hline 
  \emph{True} & Yes & \texttt{TRUE} & k+1  \\
  \emph{True} & No & \texttt{FALSE} & NA  \\
  $\varphi$ & Yes & \texttt{TRUE} & min(k) : $\varphi$  \\
  $\varphi$ & No & \texttt{FALSE} & NA  \\
\end{tabular}
\label{tab:simulation_results}
\caption{Simulation results}
\end{table}

\begin{example}
  Running \texttt{CoSA -i examples/counters/counters.sts --simulate}
  will generate a system execution of length equal to the default
  setting for the parameter \texttt{-k}, which is 10.
\end{example}

\subsection{Safety and LTL verification}

Safety (parameter \texttt{--safety}) and LTL (parameter
\texttt{--ltl}) verifications require to provide a property using the
parameter \texttt{-p}. An additional parameter \texttt{--prove}
requires the tool to try to prove that the property holds,
independently from the depth of the analysis (i.e.,
\texttt{--bmc\_length}/\texttt{-k} parameter). The result of the
analysis can be either \texttt{FALSE}, \texttt{TRUE}, or
\texttt{UNKNOWN}. The latter is provided when only a bounded proof
exists, and the probability to find an unbounded proof is dependent
from the depth of the analysis. Table~\ref{tab:safety_results}
summarizes the possible results according with the parameter
\texttt{--prove}, and the existence of a proof.

\begin{table}[h]
  \centering
\begin{tabular}{ c c c c | c c }
  Property & Prove & Trace exists? & Proof found? & Result & Trace length \\ \hline
  $\varphi$ & Yes & Yes & NA & \texttt{FALSE} & min(k) : $\varphi$  \\
  $\varphi$ & Yes & No & Yes & \texttt{TRUE} & NA  \\
  $\varphi$ & Yes & No & No & \texttt{UNKNOWN} & NA  \\ \\
  $\varphi$ & No & Yes & NA & \texttt{FALSE} & min(k) : $\varphi$  \\
  $\varphi$ & No & No & NA & \texttt{UNKNOWN} & NA  \\
\end{tabular}
\label{tab:safety_results}
\caption{Safety and LTL results}
\end{table}

\begin{example}
  Running \texttt{CoSA -i examples/counters/counters.json --add-clock
    --safety\\ -p "count1.r.reg0.out < 19\_16"} performs a safety
  analysis of the property \texttt{count1.r.reg0.out < 19\_16}, and
  the result is shown in Figure~\ref{safety_unk}. While adding the
  parameter \texttt{--prove} to the command, CoSA is able to prove the
  property and it provides the result shown in Figure~\ref{safety_true}.

\begin{lstlisting}[frame=single,language=ets,caption=Safety example (UNKNOWN),label=safety_unk]
** Problem safety **
Result: UNKNOWN
BMC depth: 10
\end{lstlisting}

\begin{lstlisting}[frame=single,language=ets,caption=Safety example (TRUE),label=safety_true]
** Problem safety **
Result: TRUE
\end{lstlisting}

\end{example}


\subsection{Equivalence checking}

The equivalence checking takes two systems and it verifies if they
start from the same initial state, and if they always receive the same
inputs they will always provide the same output. This verification
requires that the two systems have the same interface. The results of
this analysis is similar to the safety and LTL verification, and are
reported in Table~\ref{tab:equivalence_results}.

\begin{table}[h]
  \centering
\begin{tabular}{ c c c | c c }
  Prove & Trace exists? & Proof found? & Result & Trace length \\ \hline
  Yes & Yes & NA & \texttt{FALSE} & min(k) : $sys_1 \neq sys_2$  \\
  Yes & No & Yes & \texttt{TRUE} & NA  \\
  Yes & No & No & \texttt{UNKNOWN} & NA  \\ \\
  No & Yes & NA & \texttt{FALSE} & min(k) : $sys_1 \neq sys_2$  \\
  No & No & NA & \texttt{UNKNOWN} & NA  \\
\end{tabular}
\label{tab:equivalence_results}
\caption{Equivalence results}
\end{table}


\begin{example}
  Running \texttt{CoSA -i examples/mul\_2/mul\_2.json --equivalence
    examples/mul\_2/mul\_2\_pe.json --prove -k 15} performs an
  equivalence checking between \texttt{examples/mul\_2/mul\_2.json}
  and \texttt{examples/mul\_2/mul\_2\_pe.json}, and the result is
  shown in Figure~\ref{equivalence_unk}.

\begin{lstlisting}[frame=single,language=ets,caption=Equivalence example (UNKNOWN),label=equivalence_unk]
** Problem equivalence **
Result: UNKNOWN
BMC depth: 15
\end{lstlisting}

\end{example}

\subsection{Parametric model checking}


\begin{example}
  Running \texttt{CoSA -i
    examples/counters\_4/counters\_4.v[Counters\_4],examples/counters\_4/
    rst\_beh.ets
    --add-clock --safety -p "(reset\_performed \& ! posedge(rst) \&
    posedge(clk)) -> \\((next(out) < (out + 2\_16)))" -a
    "reset\_performed -> (rst = 0\_1);out < 240\_16" --prove --prefix trace}
  performs a safety analysis over the given model, property, and
  assumption, and the result is shown in Figure~\ref{parametric_safety_res}.

\begin{lstlisting}[frame=single,language=ets,caption=Safety analysis example,label=parametric_safety_res]
** Problem safety **
Result: True
\end{lstlisting}

\end{example}

\begin{example}

  Running \texttt{CoSA -i
    examples/counters\_4/counters\_4.v[Counters\_4],examples/counters\_4/
    rst\_beh.ets
    --add-clock --model-extension High --parametric -p
    "(reset\_performed \& ! posedge(rst) \& posedge(clk)) ->\\
    ((next(out) < (out + 2\_16)))" -a "reset\_performed -> (rst =
    0\_1);out < 240\_16" --prefix trace} performs a parametric
  model checking analysis extending the model with a \texttt{high}
  faulty behavior. The result is shown in
  Figure~\ref{parametric_res}.

\begin{lstlisting}[frame=single,language=ets,caption=Parametric analysis example,label=parametric_res]
** Problem parametric **
Result: UNKNOWN
BMC depth: 10
Region:
 - (counter_clk.out$FAILURE$ & counter_2.out$FAILURE$) or 
 - (counter_clk.out$FAILURE$ & counter_4.out$FAILURE$) or 
 - (counter_clk.out$FAILURE$ & counter_3.out$FAILURE$)
Executions: [1], [2], [3]
Traces (max) length: 4

*** TRACES ***

[1]:	trace[1]-parametric.txt
[2]:	trace[2]-parametric.txt
[3]:	trace[3]-parametric.txt
\end{lstlisting}

\end{example}


\section{Problem file}
\label{sec:problem_file}


\section{Results analysis}
\label{sec:results_analysis}

\subsection{Counterexample traces}

The result of the verification can in some cases provide a
counterexample trace, which is an execution of the system that
supports the result of the analysis. CoSA produces finite-state
traces, but in case of LTL verification the trace can be infinite with
a lasso-shaped loop. CoSA can generate traces either in a human
readable textual format or in VCD format.



\subsubsection{Finite traces}

The format used by CoSA to represent traces is similar to the ETS
format (as shown in the Figure~\ref{simulation_trace}), and it
provides a series of states with assignments to the variables. The
default configuration shows only the top level \texttt{INPUT} and
\texttt{OUTPUT} ports, and all variables can be shown by using the
parameter \texttt{--trace-all-vars}. Moreover, a trace shows only the
variables that do not change their value, and this option can be
disabled with the parameter \texttt{--trace-vars-change}.

\begin{example}
  Running \texttt{CoSA -i examples/counters\_4/counters\_4.v[Counters\_4] --simulate -k5}
  will generate a system execution of length equal to 6 (5 steps + initial state), as
  shown in Figure~\ref{simulation_trace}.

\begin{lstlisting}[frame=single,language=ets,caption=Simulation Counter\_4 ,label=simulation_trace]
** Problem simulation **
Result: TRUE
Execution:
---> INIT <---
  I: clk = 1_1
  I: out = 8_16
  I: rst = 1_1

---> STATE 1 <---

---> STATE 2 <---

---> STATE 3 <---

---> STATE 4 <---
  S4: clk = 0_1
  S4: rst = 0_1

---> STATE 5 <---
  S5: out = 0_16
  S5: rst = 1_1
\end{lstlisting}

\end{example}

\begin{example}
  Running \texttt{CoSA -i
    examples/counters\_4/counters\_4.v[Counters\_4] --simulate -k5\\
    --trace-vars-change} will generate a system execution of length
  equal to 6 (5 steps + initial state), showing also the variables
  that change value, as shown in Figure~\ref{simulation_trace_change}.

\begin{lstlisting}[frame=single,language=ets,caption=Simulation Counter\_4 (with changing values),label=simulation_trace_change]
** Problem simulation **
Result: TRUE
Execution:
---> INIT <---
  I: clk = 1_1
  I: out = 8_16
  I: rst = 1_1

---> STATE 1 <---
  S1: clk = 1_1
  S1: out = 8_16
  S1: rst = 1_1

---> STATE 2 <---
  S2: clk = 1_1
  S2: out = 8_16
  S2: rst = 1_1

---> STATE 3 <---
  S3: clk = 1_1
  S3: out = 8_16
  S3: rst = 1_1

---> STATE 4 <---
  S4: clk = 0_1
  S4: out = 8_16
  S4: rst = 0_1

---> STATE 5 <---
  S5: clk = 1_1
  S5: out = 9_16
  S5: rst = 0_1
\end{lstlisting}

\end{example}


We can notice that state 5 differs between traces in Figures
\ref{simulation_trace} and \ref{simulation_trace_change}. In fact CoSA
does not guarantee a deterministic result in terms because the
internal solver can produce different models according with the state
space exploration that has been performed.

The parameter \texttt{--prefix} allows the user to save the trace to
file.

\begin{example}
  Running \texttt{CoSA -i
    examples/counters\_4/counters\_4.v[Counters\_4] --simulate -k5
    --vcd --prefix trace} will generate a system execution of length
  equal to 6, and save it to a file with prefix ``trace'', as shown in
  Figure~\ref{simulation_trace_prefix}. A VCD trace is also generated
  when \texttt{--vcd} is set.

\begin{lstlisting}[frame=single,language=ets,caption=Simulation Counter\_4 (with prefix),label=simulation_trace_prefix]
** Problem simulation **
Result: TRUE
Executions: [1], [2]
Traces (max) length: 6

*** TRACES ***

[1]:	trace[1]-simulation.txt
[2]:	trace[2]-simulation.vcd
\end{lstlisting}

\end{example}



\subsubsection{Infinite traces}

The counterexample showing the violation of an LTL property usually
requires an infinite trace. For instance, if the property $F(\varphi)$
does not hold it means that the system can start in $\neg \varphi$,
and never reach a state where $\varphi$ holds. A trace that will show
this behavior is usually represented with a lasso-shaped loop. An
example is shown in Figure~\ref{ltl_trace}, where the system can loop
forever from \texttt{STATE 1} to \texttt{INIT} to \texttt{STATE 1},
and so on.

\begin{example}
  Running \texttt{CoSA -i
    examples/counter/counter.json,examples/counter/counter\_live.sts
    --ltl -a "en\_clr = 0\_1 -> self.clr = 0\_1" -p "F(self.out =
    4\_16)" --trace-vars-change} will generate a counterexample trace
  for the property \texttt{F(self.out = 4\_16)}, as shown in
  Figure~\ref{ltl_trace}.

\begin{lstlisting}[frame=single,language=ets,caption=Simulation Counter\_4 (with changing values),label=ltl_trace]
** Problem ltl **
Result: FALSE
Counterexample:
---> INIT <---
  I: self.clk = 1_1
  I: self.clr = 0_1
  I: self.out = 0_16

---> STATE 1 <---
  S1: self.clk = 1_1
  S1: self.clr = 0_1
  S1: self.out = 0_16

---> INIT (Loop) <---
\end{lstlisting}

\end{example}


\section{Good practice}

Gradually increase system complexity

\begin{itemize}
\item bottom-up verification
\item simple properties
\item awareness of property semantics
\end{itemize}


\section{Encodings}
\label{sec:encodings}


\section{Optimizations}
\label{sec:optimizations}

\subsection{Multi processing}

\subsection{Clock abstraction}

\subsection{Assume if true}

\subsection{Lemmas}

\subsection{Cone of Influence}

\subsection{Files Caching}


\section{Debugging}
\label{sec:debugging}

\newpage
\bibliographystyle{abbrv}
\bibliography{refs}

\end{document}
